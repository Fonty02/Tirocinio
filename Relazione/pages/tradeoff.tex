\subsection{Trade-off}


\subsubsection{Introduzione}
In questa sezione verranno analizzati i trade-off tra le varie metriche di valutazione e le emissioni di CO2 analizzando un dataset per volta.

Di seguito un elenco delle metriche con una piccola descrizione:
\begin{itemize}
    \item Recall: è una metrica che misura la capacità di un modello di raccomandare gli item rilevanti per un utente
    \item NDCG: è una metrica che misura la qualità delle raccomandazioni.
    \item Gini Index: è una metrica che misura l'equità nella distribuzione delle raccomandazioni. Un valore più vicino a zero indica una distribuzione più equa
    \item Average Popularity: è una metrica che misura la popolarità media degli item raccomandati. Un valore alto indica che le raccomandazioni sono concentrate su item popolari.
\end{itemize}

\subsubsection{LFM-1b\_artist\_20U50I}


\begin{table}[H]
    \centering
    \footnotesize
    \setlength\tabcolsep{0pt}
    \begin{tabularx}{\textwidth}{|X|X|}
        \hline
        \includegraphics[width=\linewidth, trim=0 0 0 0]{images/recall@10\_LFM-1b\_artist_20U50I.png} &
        \includegraphics[width=\linewidth, trim=0 0 0 0]{images/ndcg@10\_LFM-1b\_artist_20U50I.png} \\
        \hline
        \includegraphics[width=\linewidth, trim=0 0 0 0]{images/giniindex@10\_LFM-1b\_artist_20U50I.png} &
        \includegraphics[width=\linewidth, trim=0 0 0 0]{images/averagepopularity@10\_LFM-1b\_artist_20U50I.png} \\
        \hline
    \end{tabularx}
    \caption{Trade-off con il dataset LFM-1b\_artist\_20U50I}
    \label{tab:emissions_info}
\end{table}

\noindent Come già visto precedentemente, DGCF è il modello che emette di più. Nonostante ciò possiamo notare che per la recall e l'ndcg le sue performance risultano peggiori rispetto ad algoritmi più semplici come l'ItemKNN che risulta essere uno degli algoritmi che emette meno e performa meglio in queste metriche.
Per quanto riguarda il Gini Index possiamo notare che DGCF si comporta meglio di molti altri modelli ma l'ItemKNN e LINE risultano essere migliori di quest'ultimo. LINE è il miglior algoritmo.
Infine, per quanto riguarda l'Average Popularity, anche in questo caso possiamo notare anche che DGCF performa meglio di altri modelli, ma LINE risulta il miglior in assoluto ed è uno degli algoritmi che emette meno.


\subsubsection{LFM-1b\_artist\_20U50I\_75strat}


\begin{table}[H]
    \centering
    \footnotesize
    \setlength\tabcolsep{0pt}
    \begin{tabularx}{\textwidth}{|X|X|}
        \hline
        \includegraphics[width=\linewidth, trim=0 0 0 0]{images/recall@10\_LFM-1b\_artist_20U50I\_75strat.png} &
        \includegraphics[width=\linewidth, trim=0 0 0 0]{images/ndcg@10\_LFM-1b\_artist_20U50I\_75strat.png} \\
        \hline
        \includegraphics[width=\linewidth, trim=0 0 0 0]{images/giniindex@10\_LFM-1b\_artist_20U50I\_75strat.png} &
        \includegraphics[width=\linewidth, trim=0 0 0 0]{images/averagepopularity@10\_LFM-1b\_artist_20U50I\_75strat.png} \\
        \hline
    \end{tabularx}
    \caption{Trade-off con il dataset LFM-1b\_artist\_20U50I}
    \label{tab:emissions_info}
\end{table}


\noindent Come già visto precedentemente, DGCF è il modello che emette di più. Nonostante ciò possiamo notare che per la recall e l'ndcg le sue performance risultano peggiori rispetto ad algoritmi più semplici come l'ItemKNN che risulta essere uno degli algoritmi che emette meno e performa meglio in queste metriche.
Per quanto riguarda il Gini Index possiamo notare che DGCF si comporta meglio di molti altri modelli ma l'ItemKNN e LINE risultano essere migliori di quest'ultimo. ItemKNN è il miglior algoritmo.
Infine, per quanto riguarda l'Average Popularity, anche in questo caso possiamo notare anche che DGCF performa meglio di altri modelli, ma LINE risulta il miglior in assoluto ed è uno degli algoritmi che emette meno.





\subsubsection{LFM-1b\_artist\_20U50I\_50strat}


\begin{table}[H]
    \centering
    \footnotesize
    \setlength\tabcolsep{0pt}
    \begin{tabularx}{\textwidth}{|X|X|}
        \hline
        \includegraphics[width=\linewidth, trim=0 0 0 0]{images/recall@10\_LFM-1b\_artist_20U50I\_50strat.png} &
        \includegraphics[width=\linewidth, trim=0 0 0 0]{images/ndcg@10\_LFM-1b\_artist_20U50I\_50strat.png} \\
        \hline
        \includegraphics[width=\linewidth, trim=0 0 0 0]{images/giniindex@10\_LFM-1b\_artist_20U50I\_50strat.png} &
        \includegraphics[width=\linewidth, trim=0 0 0 0]{images/averagepopularity@10\_LFM-1b\_artist_20U50I\_50strat.png} \\
        \hline
    \end{tabularx}
    \caption{Trade-off con il dataset LFM-1b\_artist\_20U50I}
    \label{tab:emissions_info}
\end{table}


\noindent Come già visto precedentemente, DGCF è il modello che emette di più. Nonostante ciò possiamo notare che per la recall e l'ndcg le sue performance risultano peggiori rispetto ad altri algoritmi che emettono meno come CKE e CKFG(anch'essi di tipo Knowledge-Aware)..
Per quanto riguarda il Gini Index possiamo notare che DGCF si comporta meglio di molti altri modelli ma l'ItemKNN risulta essere migliore di quest'ultimo ed il migliore in assoluto.
Infine, per quanto riguarda l'Average Popularity, anche in questo caso possiamo notare anche che DGCF performa meglio di altri modelli, ma LINE risulta il miglior in assoluto ed è uno degli algoritmi che emette meno.



\subsubsection{LFM-1b\_artist\_20U50I\_25strat}


\begin{table}[H]
    \centering
    \footnotesize
    \setlength\tabcolsep{0pt}
    \begin{tabularx}{\textwidth}{|X|X|}
        \hline
        \includegraphics[width=\linewidth, trim=0 0 0 0]{images/recall@10\_LFM-1b\_artist_20U50I\_25strat.png} &
        \includegraphics[width=\linewidth, trim=0 0 0 0]{images/ndcg@10\_LFM-1b\_artist_20U50I\_25strat.png} \\
        \hline
        \includegraphics[width=\linewidth, trim=0 0 0 0]{images/giniindex@10\_LFM-1b\_artist_20U50I\_25strat.png} &
        \includegraphics[width=\linewidth, trim=0 0 0 0]{images/averagepopularity@10\_LFM-1b\_artist_20U50I\_25strat.png} \\
        \hline
    \end{tabularx}
    \caption{Trade-off con il dataset LFM-1b\_artist\_20U50I}
    \label{tab:emissions_info}
\end{table}


\noindent Come già visto precedentemente, DGCF è il modello che emette di più. Nonostante ciò possiamo notare che per la recall e l'ndcg le sue performance risultano peggiori rispetto ad altri algoritmi che emettono meno come CKE e CKFG (anch'essi di tipo Knowledge-Aware).
Per quanto riguarda il Gini Index possiamo notare che DGCF si comporta meglio di molti altri modelli ma l'ItemKNN risulta essere di quest'ultimo migliore ed il migliore in assoluto.
Infine, per quanto riguarda l'Average Popularity, in questo caso possiamo notare anche che DGCF è uno dei peggiori mentre ItemKNN risulta il miglior in assoluto ed è l'algoritmo che emette meno.


\subsubsection{ml-10m\_50U10I}


\begin{table}[H]
    \centering
    \footnotesize
    \setlength\tabcolsep{0pt}
    \begin{tabularx}{\textwidth}{|X|X|}
        \hline
        \includegraphics[width=\linewidth, trim=0 0 0 0]{images/recall@10_ml-10m_50U10I.png} &
        \includegraphics[width=\linewidth, trim=0 0 0 0]{images/ndcg@10_ml-10m_50U10I.png} \\
        \hline
        \includegraphics[width=\linewidth, trim=0 0 0 0]{images/giniindex@10_ml-10m_50U10I.png} &
        \includegraphics[width=\linewidth, trim=0 0 0 0]{images/averagepopularity@10_ml-10m_50U10I.png} \\
        \hline
    \end{tabularx}
    \caption{Trade-off con il dataset ml-10m\_50U10I}
    \label{tab:emissions_info}
\end{table}

\noindent LightGCN è l'algoritmo che emette di più, osservando i valori da 5 a circa 180 volte rispetto agli altri modelli.
Emissioni cosi alte non sono però giustificate. Nelle metriche di recall e ndcg LightGCN è il secondo migliore, ma la differenza rispetto al primo è molto bassa e quindi non giustifica emissioni così alte.
ItemKNN si conferma uno dei migliori algoritmi per queste due metriche in quanto emette meno ed è uno dei più performanti.
Per quanto riguarda le metriche di Gini Index e Average Popularity possiamo notare che LightGCN è uno di migliori, ma anche in questo caso la differenza rispetto ad altri modelli non giustifica emissioni così alte. ItemKNN si conferma con il secondo peggior modello come punteggi. Line risulta essere il migliore in quanto è il secondo per basse emissioni ed ottiene i punteggi migliori. 

\subsubsection{Conclusioni}

Si può facilmente notare come il trade-off emissioni-performance sia decisamente a svantaggio dell'DGCF. Infatti, a fronte di emissioni molto elevate, le performance risultato spesso essere peggiori di modelli molto più semplici.
Con i due dataset più grandi possiamo notare come in generale ItemKNN risulti essere uno degli algoritmi con il miglior trade-off emissioni-performance nelle metriche di ranking, mentre LINE risulta essere il migliore nelle metriche di popolarità e equità nelle distribuzioni.
Al diminuire della dimensione del dataset DGCF comincia a comportarsi meglio nelle metriche di popolarità e equità, ma le sue emissioni rimangono sempre molto alte e non giustificano una possibile scelta di questo modello.
ItemKNN comincia a non performare bene nelle metriche di ranking, mentre migliora nelle metriche di popolarità e equità, arrivando anche a risultare il migliore

