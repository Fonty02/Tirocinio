\section{Recommender Systems}

\subsection{Introduzione}

Un sistema di raccomandazione (Recommender System) \cite{RecommenderOverview} è un sistema software progettato per suggerire all'utente elementi di interesse, come ad esempio prodotti, servizi, informazioni o contenuti multimediali, in base alle preferenze e ai comportamenti passati dell'utente. I sistemi di raccomandazione sono ampiamente utilizzati in diversi contesti, come ad esempio il commercio elettronico, i social network, i servizi di streaming multimediale e le piattaforme di ricerca e informazione. I sistemi di raccomandazione sono utili per migliorare l'esperienza dell'utente, aumentare la soddisfazione e la fidelizzazione del cliente, e favorire la scoperta di nuovi contenuti e opportunità.
Questi sistemi sono basati su algoritmi di apprendimento automatico e intelligenza artificiale, che analizzano i dati relativi alle preferenze e ai comportamenti degli utenti, e generano raccomandazioni personalizzate in base a tali informazioni. I sistemi di raccomandazione possono essere di diversi tipi, a seconda della tecnica utilizzata per generare le raccomandazioni.
\begin{itemize}
    \item \textbf{Content-based}: i sistemi di raccomandazione content-based generano raccomandazioni basate sul contenuto degli elementi e sulle preferenze dell'utente. Questi sistemi analizzano le caratteristiche degli elementi e le preferenze dell'utente, e generano raccomandazioni in base alla somiglianza tra gli elementi e le preferenze dell'utente. Per preferenze dell'utente si intendono le caratteristiche degli elementi che l'utente ha valutato positivamente in passato.
    
    \item \textbf{Collaborative filtering}: i sistemi di raccomandazione collaborative filtering generano raccomandazioni basate sulla somiglianza tra gli utenti. Questi sistemi analizzano i dati relativi alle preferenze e ai comportamenti degli utenti, e generano raccomandazioni in base alla somiglianza tra gli utenti. I sistemi di raccomandazione collaborative filtering possono essere di due tipi: user-based e item-based. I sistemi user-based generano raccomandazioni basate sulla somiglianza tra gli utenti, mentre i sistemi item-based generano raccomandazioni basate sulla somiglianza tra gli item.
    \item \textbf{Approci ibridi}: i sistemi di raccomandazione ibridi combinano tecniche di content-based e collaborative filtering per generare raccomandazioni personalizzate.
    \item \textbf{Knowledge-based} \cite{KnowledgeBased}: i sistemi di raccomandazione knowledge-based generano raccomandazioni basate basandosi su conoscenza semantica
\end{itemize}

\noindent Per valutare le prestazioni di un sistema di raccomandazione si possono utilizzare diverse metriche che è possibile riassumere nelle seguenti categorie:
\begin{itemize}
    \item \textbf{Accuracy metrics}: queste metriche valutano la precisione e l'accuratezza delle raccomandazioni generate dal sistema. Alcune delle metriche più comuni sono l'RMSE (Root Mean Squared Error) e il MAE (Mean Absolute Error).
    \item \textbf{Ranking metrics}: queste metriche valutano la qualità dell'ordinamento delle raccomandazioni generate dal sistema. Alcune delle metriche più comuni sono il coefficiente di correlazione di Kendall Tau il coefficiente di correlazione di Spearman, l'NDCG (Normalized Discounted Cumulative Gain).
    \item \textbf{Diversity metrics}: queste metriche valutano la diversità delle raccomandazioni generate dal sistema. Alcune delle metriche più comuni sono la diversità delle raccomandazioni e la novità delle raccomandazioni.
    \item \textbf{Coverage metrics}: queste metriche valutano la copertura degli elementi raccomandati dal sistema.
    
    \item \textbf{Classification metrics}: queste metriche valutano la capacità del sistema di classificare correttamente gli elementi in base alle preferenze dell'utente. Alcune delle metriche più comuni sono l'accuracy, la precision, il recall e l'F1-score.
\end{itemize}

\noindent Alcuni tipici problemi che si possono incontrare nella progettazione e nell'implementazione di un sistema di raccomandazione sono:
\begin{itemize}
    \item \textbf{Cold start problem} \cite{ColdStart}: il problema del cold start si verifica quando un nuovo utente o un nuovo elemento si registra nel sistema e non ci sono dati sufficienti per generare raccomandazioni personalizzate.
    \item \textbf{Data sparsity problem} \cite{DataSparsity}: nella maggior parte delle reali applicazioni il numero di item è molto maggiore del numero di item valutati da ciascun utente. Questo porta a una matrice di valutazioni molto sparsa, che rende difficile la generazione di raccomandazioni accurate
    \item \textbf{Vulnerabilità agli attacchi} \cite{Attacchi} : i sistemi di raccomandazione possono essere vulnerabili a diversi tipi di attacchi, come ad esempio le recensioni fake (tipico problema degli e-commerce)
\end{itemize}

\subsection{Recommender Systems e Sustainability}

I sistemi di raccomandazione, così come tutti gli altri modelli di AI, possono essere utilizzati per promuovere la sostenibilità in tutti i suoi punti, ad esempio cercando di adempiere agli obiettivi dell'Agenda 2030 \cite{RecommenderSustainability}.

\noindent In ambito \textit{energia pulita e riduzione delle emissioni} viene suggerito come i sistemi di raccomandazione possano essere utilizzati per promuovere l'adozione di comportamenti sostenibili, ad esempio suggerendo all'utente di utilizzare mezzi di trasporto pubblici o condivisi, di ridurre il consumo di energia elettrica o di acquistare prodotti sostenibili. Inoltre, i sistemi di raccomandazione possono essere utilizzati per promuovere l'adozione di energie rinnovabili e la riduzione delle emissioni di gas serra. In questo caso si parla dunque di sistemi di raccomandazione i quali cercano di promuovere comportamenti sostenibili, ma a loro vola per essere addestrati richiedono grandi quantità di dati e di risorse computazionali, che possono avere un impatto negativo sull'ambiente.

\noindent Una soluzione dunque può essere quella di addestrare i modelli di raccomandazione in modo sostenibile, senza però perdere di performance.

\noindent
Tracciare le emissioni degli algoritmi di raccomandazione e cercare di prevederle è molto importante quando si parla di sviluppo sostenibile in campo RecSys. Ancora oggi si tende a trascurare l'impatto ambientale di un'attività e, in questo ambito, si è molto propensi nell'utilizzare dei modelli molti complessi e pesanti
che richiedono molte risorse per essere addestrati ed eseguti per ottenere delle buone performance. Spesso, però, modelli molto più leggeri e semplici riescono a ottenere delle performance molto simili (se non superiori) a modelli più complessi e il tutto con un impatto ambientale decisamente minore.
Ad oggi il carbon dioxide equivalent (CO$_2$eq) è il principale indicatore utilizzato da governi e enti per misurare l'impatto ambientale di un'attività.
Il CO$_2$eq è un'unità di misura che esprime l'equivalente in CO$_2$ di tutti i gas serra emessi da un'attività, in modo da poter confrontare l'impatto ambientale di attività diverse.
Una strategia comune per calcolare il CO$_2$eq è quella di moltiplicare tra loro il \textbf{carbon intensity(CI)} e l'\textbf{energia consumata(PC)} dall'attività (nel nostro caso l'esecuzione di algoritmi).



\begin{equation*}
    \textit{emission} = \textit{CI}  \cdot \textit{PC}
\end{equation*}

\noindent In particolare i valori di CI dipendono dalle diverse fonti di energia utilizzate durante la computazione 
(es. energia solare, energia eolica, etc.). Se \textit{s} è la fonte di energia,  \textit{e$_s$} sono le emissioni per KW/h di energia e \textit{p$_s$}  è la percentuale di energia prodotta dalla fonte s, allora il CI è dato da:
\begin{equation*}
    \textit{CI} = \sum_{s \in S} \textit{e$_s$} \cdot \textit{p$_s$}
\end{equation*}


