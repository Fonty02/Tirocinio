\section{Configurazione}

\subsection{Parametri di running}
Qui di seguito vengono riportati i parametri di running dei vari esperimenti effettuati.
\subsubsection*{Parametri di environment}
I parametri di environment servono per configurare l'ambiente di esecuzione.
\begin{itemize}
    \item \textbf{gpu\_id}: 0
    \item \textbf{worker}: 0
    \item \textbf{use\_gpu}: True
    \item \textbf{seed}: 2020
    \item \textbf{state}: INFO
    \item \textbf{encoding}: utf-8
    \item \textbf{reproducibility}: True
    \item \textbf{shuffle}: True
\end{itemize}

\subsubsection*{Parametri di training}
I parametri di training servono per l'addestramento dei modelli.
\begin{itemize}
    \item \textbf{epochs}: 200
    \item \textbf{train\_batch\_size}: 2048
    \item \textbf{learner}: adam
    \item \textbf{learning\_rate}: .001
    \item \textbf{train\_neg\_sample\_args}: 
    \begin{itemize}
        \item \textbf{distribution}: uniform
        \item \textbf{sample\_num}: 1
        \item \textbf{dynamic}: False
        \item \textbf{candidate\_num}: 0
    \end{itemize}
    \item \textbf{eval\_step}: 1
    \item \textbf{stopping\_step}: 10
    \item \textbf{clip\_grad\_norm}: None
    \item \textbf{loss\_decimal\_place}: 4
    \item \textbf{weight\_decay}: .0
    \item \textbf{require\_pow}: False
    \item \textbf{enable\_amp}: False
    \item \textbf{enable\_scaler}: False
\end{itemize}

\subsubsection*{Parametri di evaluation}
I parametri di evaluation servono per valutare i modelli.
\begin{itemize}
    \item \textbf{eval\_args}:
    \item \begin{itemize}
                \item \textbf{group\_by}: user
                \item \textbf{order}: RO
                \item \textbf{split}: RS : [0.8, 0.1, 0.1]
                \item \textbf{mode}: full
            \end{itemize}
    \item \textbf{repeatable}: False
    \item \textbf{metrics}: ['Recall', 'MRR', 'NDCG', 'Hit', 'MAP', 'Precision', 'GAUC', 'ItemCoverage', 'AveragePopularity', 'GiniIndex', 'ShannonEntropy', 'TailPercentage']
    \item \textbf{topk}: 10
    \item \textbf{valid\_metric}: MRR@10
    \item \textbf{eval\_batch\_size}: 4096
    \item \textbf{metric\_decimal\_place}: 4
\end{itemize}

\subsubsection*{Iper parametri dei modelli}
Gli iper parametri dei modelli sono un insieme di parametri che vengono utilizzati per configurare i modelli. La loro configurazione può influenzare il risultato finale. Esistono delle tecniche di HyperTuning che permettono di trovare i migliori iper parametri per un determinato modello e dataset.
In questo caso si è scelto di utilizzare gli iper parametri di default

