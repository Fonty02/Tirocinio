\section{Abstract}

Lo scopo di questa tesi è quello di analizzare la sostenibilità ambientale dei sistemi di raccomandazione, in particolare di quelli basati su algoritmi di apprendimento automatico.\\
Si vuole indagare su quello che è il trade-off tra le performance dei modelli di raccomandazione a stato dell'arte e il loro impatto ambientale e vedere se sia possibile ridurre quest'ultimo senza compromettere in modo significativo le performance.\\
Per fare ciò si è scelto di addestrare alcuni modelli di raccomandazione a stato dell'arte su dataset di dimensioni diverse e cercare di capire il trade-off tra performance e impatto ambientale di ciascuno di essi. Successivamente si è cercato di ridurre l'impatto ambientale di questi modelli lavorando sul criterio di early stopping, cercando di capire se sia possibile utilizzare un criterio di early stopping basato anche sulle emissioni dei modelli. Anche questi esperimenti sono stati condotti su dataset di dimensioni diverse e sui diversi modelli di raccomandazione a stato dell'arte per valutare il trade-off tra performance e impatto ambientale. Successivamente si sono confrontati i risultati ottenuti con quelli precedenti. \\I risultati ottenuti mostrano che, in generale, è possibile ridurre l'impatto ambientale dei modelli di raccomandazione senza compromettere in modo significativo le performance dei modelli stessi.