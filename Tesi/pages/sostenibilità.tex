\section{Sostenibilità}

\subsection{Introduzione alla sostenibilità}

La sostenibilità è un concetto che si è diffuso negli ultimi anni, in particolare a partire dagli anni '80, e che ha assunto un ruolo sempre più importante nella società contemporanea. La sostenibilità è un concetto complesso e multidimensionale, che riguarda diversi aspetti della vita umana e dell'ambiente in cui viviamo. In generale, la sostenibilità si riferisce alla capacità di soddisfare i bisogni delle generazioni presenti senza compromettere la capacità delle generazioni future di soddisfare i propri bisogni. Questo significa che la sostenibilità riguarda la gestione responsabile delle risorse naturali, la tutela dell'ambiente, la promozione dello sviluppo economico e sociale, e la garanzia di un futuro migliore per tutti.
Un'esempio concreto di impegno per la sostenibilità è l' Agenda 2030 per lo Sviluppo Sostenibile, adottata dalle Nazioni Unite nel 2015. L'Agenda 2030 è un piano d'azione globale che contiene 17 Obiettivi di Sviluppo Sostenibile (SDGs) e 169 obiettivi specifici, che coprono una vasta gamma di temi, tra cui la povertà, la fame, la salute, l'istruzione, l'uguaglianza di genere, l'acqua, l'energia, il clima, l'ambiente, la pace e la giustizia. L'obiettivo dell'Agenda 2030 è quello di promuovere uno sviluppo sostenibile che sia equo, inclusivo e rispettoso dell'ambiente, e di garantire che nessuno venga lasciato indietro.\\
I 17 Obiettivi di Sviluppo Sostenibile dell'Agenda 2030 sono i seguenti:
\begin{enumerate}
    \item Fine della povertà: porre fine alla povertà in tutte le sue forme ovunque
    \item Fame zero: porre fine alla fame, raggiungere la sicurezza alimentare e migliorare la nutrizione e promuovere l'agricoltura sostenibile
    \item Salute e benessere: garantire una vita sana e promuovere il benessere per tutti in ogni età
    \item Istruzione di qualità: garantire un'istruzione inclusiva, equa e di qualità e promuovere opportunità di apprendimento permanente per tutti
    \item Parità di genere: raggiungere l'uguaglianza di genere e l'empowerment di tutte le donne e le ragazze
    \item Acqua pulita e igiene: garantire la disponibilità e la gestione sostenibile dell'acqua e dell'igiene per tutti
    \item \textbf{Energia pulita e accessibile} : garantire l'accesso a un'energia affidabile, sostenibile e moderna per tutti
    \item Lavoro dignitoso e crescita economica : promuovere la crescita economica sostenuta, inclusiva e sostenibile, il pieno e produttivo impiego e il lavoro dignitoso per tutti
    \item \textbf{Industria, innovazione e infrastrutture}: costruire infrastrutture resilienti, promuovere l'industrializzazione sostenibile e promuovere l'innovazione
    \item Riduzione delle disuguaglianze: ridurre le disuguaglianze all'interno e tra i paesi
    \item \textbf{Città e comunità sostenibili}: rendere le città e gli insediamenti umani inclusivi, sicuri, resilienti e sostenibili
    \item \textbf{Consumo e produzione responsabili}: garantire modelli di consumo e produzione sostenibili
    \item \textbf{Azione per il clima}: adottare misure urgenti per combattere il cambiamento climatico e i suoi impatti
    \item Vita sott'acqua: conservare e utilizzare in modo sostenibile gli oceani, i mari e le risorse marine per uno sviluppo sostenibile
    \item Vita sulla terra: proteggere, ripristinare e promuovere l'uso sostenibile dell'ecosistema terrestre, gestire in modo sostenibile le foreste, combattere la desertificazione, arrestare e invertire il degrado del suolo e arrestare la perdita di biodiversità
    \item Pace, giustizia e istituzioni solide: promuovere società pacifiche e inclusive per lo sviluppo sostenibile, fornire accesso alla giustizia per tutti e costruire istituzioni efficaci, responsabili e inclusive a tutti i livelli
    \item Partenariati per gli obiettivi: rafforzare i mezzi di attuazione e rinnovare il partenariato globale per lo sviluppo sostenibile
\end{enumerate}


\begin{figure}[H]
    \centering
    \includegraphics[scale=0.5]{images/sdg.png}
\caption{I 17 Obiettivi di Sviluppo Sostenibile dell'Agenda 2030}
\end{figure}
\noindent In ambito di Intelligence Artificiale, la sostenibilità è un tema di grande rilevanza, in quanto l'uso di tecnologie avanzate come l'AI può avere un impatto significativo sull'ambiente e sulla società. Ad esempio, l'AI richiede una grande quantità di risorse energetiche per funzionare, e può avere un impatto negativo sull'ambiente se non viene utilizzata in modo responsabile. Inoltre, l'AI può avere effetti sociali indesiderati, come la discriminazione e l'esclusione di determinati gruppi di persone. Per questo motivo, è importante che lo sviluppo e l'uso dell'AI siano guidati dai principi della sostenibilità, al fine di garantire che l'AI contribuisca a uno sviluppo sostenibile e equo per tutti.


\subsection{Sostenibilità ambientale}
La sostenibilità ambientale è uno degli aspetti più importanti della sostenibilità, in quanto l'ambiente è la base su cui si fonda la vita umana e la prosperità economica. La sostenibilità ambientale riguarda la gestione responsabile delle risorse naturali, la tutela dell'ambiente e la prevenzione dell'inquinamento e del degrado ambientale. La sostenibilità ambientale si basa su principi come il rispetto per la natura, la conservazione della biodiversità, la riduzione delle emissioni di gas serra e la promozione di energie rinnovabili e pulite. La sostenibilità ambientale è fondamentale per garantire un futuro sostenibile per tutti, e per preservare il pianeta per le generazioni future.


\subsection{Green AI}

In ambito Intelligenza Artificiale e sostenibilità ambientale possiamo distingure due tipi di AI sostenibile \cite{sostenibilita}: \textit{Sustainability of AI} e \textit{AI for Sustainability}. Il primo ramo ha come obiettivo quello di quello di misurare la sostenibilità dello sviluppo e dell'uso di modelli AI, ad esempio misurando la \textit{carbon footprint} e l'energia usata per addestrare un modello. Il secondo ramo, invece, si occupa di utilizzare l'AI per affrontare le sfide della sostenibilità, ad esempio sviluppando modelli per la previsione del cambiamento climatico o per la gestione delle risorse naturali.

\noindent Con il termine \textcolor{green}{Green AI} \cite{GreenAI} ci si riferisce alla ricerca e allo sviluppo di modelli di intelligenza artificiale che tengano conto del costo computazione, delle risorse utilizzate e dell'impatto ambientale. Questa si differenzia dalla \textcolor{red}{Red AI} il cui obiettivo è quello di ottenere modelli sempre più complessi e performanti, senza tenere conto delle risorse utilizzate e dell'impatto ambientale. Dunque la Green AI promuove un approcio in cui si tiene conto del trade-off tra performance e efficienza, cercando di ottenere modelli che siano performanti ma che allo stesso tempo siano sostenibili e rispettosi dell'ambiente.
La Green AI tieni dunque conto dell' \textbf{efficienza energetica}: i modelli di AI devono essere progettati in modo da utilizzare meno risorse energetiche possibili, ad esempio riducendo il numero di parametri, la complessità del modello e il numero di operazioni computazionali.
Alcuni fattori da tenere in considerazione sono:
\begin{itemize}
    \item \textbf{Hardware}: l'hardware utilizzato per addestrare e eseguire i modelli di AI può avere un impatto significativo sull'efficienza energetica. Ad esempio, l'uso di hardware specializzato come le GPU può ridurre il tempo di addestramento e il consumo energetico rispetto all'uso di CPU tradizionali.
    \item \textbf{Algoritmi}: la scelta degli algoritmi di AI può influenzare l'efficienza energetica dei modelli. Alcuni algoritmi sono più efficienti di altri in termini di consumo energetico e risorse computazionali, e possono essere preferiti per ridurre l'impatto ambientale.
    \item \textbf{Parametri}: il numero di parametri di un modello di AI può influenzare il consumo energetico e le risorse computazionali richieste per addestrare e eseguire il modello. Ridurre il numero di parametri può migliorare l'efficienza energetica del modello.
    \item \textbf{Emissioni di gas serra}: le emissioni di gas serra prodotte durante l'addestramento e l'esecuzione dei modelli di AI possono contribuire al cambiamento climatico e all'inquinamento atmosferico. Ridurre le emissioni di gas serra è un obiettivo importante per la Green AI.
\end{itemize}


