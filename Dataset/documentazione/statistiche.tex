
\noindent LFM-1b\_artist è un dataset contenente informazioni riguardanti le interazioni tra utenti e artisti musicali.

\section*{DETASET ORIGINALE}

\subsection*{STATISTICHE DATASET}


Descrizione del dataset
\begin{table}[H]
    \centering
    \begin{tabularx}{\textwidth}{|c|X|}
        \hline
        \textbf{Feature} & \textbf{Descrizione} \\
        \hline
        n\_users & 120322 \\
        \hline
        n\_items & 3123496 \\
        \hline
        n\_inter & 65133026 \\
        \hline
        sparsity & 0.9998266933373666 \\
        \hline
    \end{tabularx}
    \caption{Informazioni sul dataset LFM-1b\_artist}
    \label{tab:dataset_info}
\end{table}


\noindent Descrizione del knowledge graph
\begin{table}[H]
    \centering
    \begin{tabularx}{\textwidth}{|c|X|}
        \hline
        n\_ent\_head & 823213 \\
        \hline
        n\_ent\_tail & 353607 \\
        \hline
        n\_rel & 8 \\
        \hline
        n\_triple & 2114049 \\
        \hline
    \end{tabularx}
    \caption{Informazioni sul knowledge graph del dataset LFM-1b\_artist}
    \label{tab:dataset_info}
\end{table}

\newpage
\noindent I nomi delle relazioni presenti nel knowledge graph sono i seguenti:
\begin{itemize}
    \item music.recording.artist
    \item music.recording.releases
    \item music.recording.producer
    \item music.recording.engineer
    \item music.recording.featured\_artists
    \item music.featured\_artist.recordings
    \item music.release.artist
    \item music.artist.release
\end{itemize}

\section*{DATASET PROCESSATO}

\subsection*{DESCRIZIONE}

\noindent Il dataset originale risultava essere troppo grande per le risorse a nostra disposizione, dunque è stato opportunamente processato. In paritcolare sono state svolte le seguenti operazioni
\begin{itemize}
    \item \textbf{Filtraggio:} il dataset è stato filtrato eliminando tutte le interazioni in cui erano coinvolti utenti e/o item con meno di 5 interazioni
    \item \textbf{Sampling:} dopo la fase di filtraggio, è stato effettuato un sampling casuale il cui scopo era quello di ridurre il numero di interazioni. In particolare sono stati selezionati casualmente 20000 utenti e 50000 item e sono state mantenute solo le interazioni in cui erano coinvolti utenti e gli item selezionati
\end{itemize}

\subsection*{STATISTICHE DATASET}


Descrizione del dataset
\begin{table}[H]
    \centering
    \begin{tabularx}{\textwidth}{|c|X|}
        \hline
        \textbf{Feature} & \textbf{Descrizione} \\
        \hline
        n\_users & 19481 \\
        \hline
        n\_items & 42547 \\
        \hline
        n\_inter & 900212 \\
        \hline
        sparsity & 0.0.9989313587429705 \\
        \hline
    \end{tabularx}
    \caption{Informazioni sul dataset LFM-1b\_artist processato}
    \label{tab:dataset_info}
\end{table}


\noindent Descrizione del knowledge graph
\begin{table}[H]
    \centering
    \begin{tabularx}{\textwidth}{|c|X|}
        \hline
        n\_ent\_head & 15509 \\
        \hline
        n\_ent\_tail & 35156 \\
        \hline
        n\_rel & 5 \\
        \hline
        n\_triple & 46827 \\
        \hline
    \end{tabularx}
    \caption{Informazioni sul knowledge graph del dataset LFM-1b\_artist processato}
    \label{tab:dataset_info}
\end{table}

\noindent I nomi delle relazioni presenti nel knowledge graph sono i seguenti:
\begin{itemize}
    \item music.recording.artist
    \item music.recording.releases
    \item music.recording.producer
    \item music.recording.engineer
    \item music.recording.featured\_artists
\end{itemize}




\section*{ESECUZIONE DEI MODELLI E RISULTATI}




