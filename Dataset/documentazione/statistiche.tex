
\noindent LFM1b\_artist è un dataset contenente informazioni riguardanti le interazioni tra utenti e artisti musicali.

\section*{DETASET ORIGINALE}

\subsection*{STATISTICHE DATASET}


Descrizione del dataset
\begin{table}[H]
    \centering
    \footnotesize
    \begin{tabularx}{\textwidth}{|c|X|}
        \hline
        \textbf{Feature} & \textbf{Descrizione} \\
        \hline
        n\_users & 120322 \\
        \hline
        n\_items & 3123496 \\
        \hline
        n\_inter & 65133026 \\
        \hline
        sparsity & 0.9998266933373666 \\
        \hline
    \end{tabularx}
    \caption{Informazioni sul dataset LFM1b\_artist}
    \label{tab:dataset_info}
\end{table}


\noindent Descrizione del knowledge graph
\begin{table}[H]
    \centering
    \footnotesize
    \begin{tabularx}{\textwidth}{|c|X|}
        \hline
        n\_ent\_head & 823213 \\
        \hline
        n\_ent\_tail & 353607 \\
        \hline
        n\_rel & 8 \\
        \hline
        n\_triple & 2114049 \\
        \hline
    \end{tabularx}
    \caption{Informazioni sul knowledge graph del dataset LFM1b\_artist}
    \label{tab:dataset_info}
\end{table}

\newpage
\noindent I nomi delle relazioni presenti nel knowledge graph sono i seguenti:
\begin{itemize}
    \item music.recording.artist
    \item music.recording.releases
    \item music.recording.producer
    \item music.recording.engineer
    \item music.recording.featured\_artists
    \item music.featured\_artist.recordings
    \item music.release.artist
    \item music.artist.release
\end{itemize}

\section*{DATASET PROCESSATO}

\subsection*{DESCRIZIONE}

\noindent Il dataset originale risultava essere troppo grande per le risorse a nostra disposizione, dunque è stato opportunamente processato. In paritcolare sono state svolte le seguenti operazioni
\begin{itemize}
    \item \textbf{Filtraggio:} il dataset è stato filtrato eliminando tutte le interazioni in cui erano coinvolti utenti e/o item con meno di 5 interazioni
    \item \textbf{Sampling:} dopo la fase di filtraggio, è stato effettuato un sampling casuale il cui scopo era quello di ridurre il numero di interazioni. In particolare sono stati selezionati casualmente 20000 utenti e 50000 item e sono state mantenute solo le interazioni in cui erano coinvolti utenti e gli item selezionati
\end{itemize}

\subsection*{STATISTICHE DATASET}


Descrizione del dataset
\begin{table}[H]
    \centering
    \footnotesize
    \begin{tabularx}{\textwidth}{|c|X|}
        \hline
        \textbf{Feature} & \textbf{Descrizione} \\
        \hline
        n\_users & 19481 \\
        \hline
        n\_items & 42547 \\
        \hline
        n\_inter & 900212 \\
        \hline
        sparsity & 0.0.9989313587429705 \\
        \hline
    \end{tabularx}
    \caption{Informazioni sul dataset LFM1b\_artist processato}
    \label{tab:dataset_info}
\end{table}


\noindent Descrizione del knowledge graph
\begin{table}[H]
    \centering
    \footnotesize
    \begin{tabularx}{\textwidth}{|c|X|}
        \hline
        n\_ent\_head & 15509 \\
        \hline
        n\_ent\_tail & 35156 \\
        \hline
        n\_rel & 5 \\
        \hline
        n\_triple & 46827 \\
        \hline
    \end{tabularx}
    \caption{Informazioni sul knowledge graph del dataset LFM1b\_artist processato}
    \label{tab:dataset_info}
\end{table}

\noindent I nomi delle relazioni presenti nel knowledge graph sono i seguenti:
\begin{itemize}
    \item music.recording.artist
    \item music.recording.releases
    \item music.recording.producer
    \item music.recording.engineer
    \item music.recording.featured\_artists
\end{itemize}


\newpage

\section*{ESECUZIONE DEI MODELLI E RISULTATI}

I modelli di recommendation sono stati eseguiti utilizzando questi parametri:
\begin{itemize}
    \item \textbf{Numero di epoche:} 1
    \item \textbf{Train batch size:} 4096
    \item \textbf{Eval batch size:} 1024
    \item \textbf{Numero neighbors per l'ItemKNN:} 20
\end{itemize}


Qui di seguito sono riportati i risultati ottenuti dai modelli di recommendation.
\begin{table}[H]
    \centering
    \footnotesize
    \begin{tabularx}{\textwidth}{|c|c|X|}
        \hline
        \textbf{Nome modello} & \textbf{Esito} & \textbf{Motivo fallimento} \\
        \hline
        ItemKNN & Successo &  \\
        \hline
        Pop & Successo &  \\
        \hline
        Random & Fallito &  Expected all tensors to be on the same device, but found at least two devices, cuda:0 and cpu! (when checking argument for argument tensors in method wrapper\_CUDA\_cat) \\
        \hline
        Simplex & Successo &  \\
        \hline
        ADMMSLIM & Fallito & Terminato in modo anomalo \\
        \hline
        BPR & Successo &  \\
        \hline
        DMF & Successo &  \\
        \hline
        ENMF & Successo &  \\
        \hline
        FISM & Successo &  \\
        \hline
        NCEPLRec & Successo &  \\
        \hline
        SLIMElasticNet & Fallito &  Expected all tensors to be on the same device, but found at least two devices, cuda:0 and cpu! (when checking argument for argument tensors in method wrapper\_CUDA\_cat) \\
        \hline
        CDAE & Successo &  \\
        \hline
        COnvNCF & Successo &  \\
        \hline
        DiffRec & Successo &  \\
        \hline
        EASE & Fallito & Terminato in modo anomalo \\
        \hline
        GCMC & Successo &  \\
        \hline
        LDiffRec & Successo &  \\
        \hline
        MacridVAE & Fallito &  CUDA out of memory. Tried to allocate 664.00 MiB \\
        \hline
        MultiDAE & Successo &  \\
        \hline
        MultiVAE & Successo &  \\
        \hline
        NAIS & Fallito & CUDA out of memory. Tried to allocate 3.97 GiB \\
        \hline
        NeuMF & Successo &  \\
        \hline
        NGCF & Successo &  \\
        \hline
        NNCF & Successo &  \\
        \hline
        LightGCN & Successo &  \\
        \hline
        RecVAE & Rimandato & Tempo di esecuzione non accettabile \\
        \hline
        DGCF & Successo &  \\
        \hline
        LINE & Successo &  \\
        \hline
        NCL & Fallito & No module faiss \\
        \hline
        SGL & Successo & \\
        \hline
        SpectralCF & Successo &  \\
        \hline
        CKE & Successo &  \\
        \hline
        CFKG & Successo &  \\
        \hline
        KGIN & Fallito & mismatch di versione tra PyTorch e torch\-scatter \\
        \hline
        KGNNLS & Successo &  \\
        \hline
        KTUP & Successo &  \\
        \hline
        MKR & Successo &  \\
        \hline
        RippleNet & Successo &  \\
        \hline
    \end{tabularx}
    \caption{Esiti di esecuzione dei modelli}
    \label{tab:dataset_info}
\end{table}


