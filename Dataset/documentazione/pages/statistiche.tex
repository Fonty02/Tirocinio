\section{Statistiche dei dataset}

LFM1b\_artist è un dataset contenente informazioni riguardanti le interazioni tra utenti e artisti musicali.



\subsection{Dataset originale}


Descrizione del dataset
\begin{table}[H]
    \centering
    \footnotesize
    \begin{tabularx}{\textwidth}{|c|X|}
        \hline
        \textbf{Feature} & \textbf{Descrizione} \\
        \hline
        n\_users & 120322 \\
        \hline
        n\_items & 3123496 \\
        \hline
        n\_inter & 65133026 \\
        \hline
        sparsity & 0.9998266933373666 \\
        \hline
        avg\_inter\_user & 541.3226675088513 \\
        \hline
    \end{tabularx}
    \caption{Informazioni sul dataset LFM1b\_artist}
    \label{tab:dataset_info}
\end{table}


\noindent Descrizione del knowledge graph
\begin{table}[H]
    \centering
    \footnotesize
    \begin{tabularx}{\textwidth}{|c|X|}
        \hline
        n\_ent\_head & 823213 \\
        \hline
        n\_ent\_tail & 353607 \\
        \hline
        n\_rel & 8 \\
        \hline
        n\_triple & 2114049 \\
        \hline
    \end{tabularx}
    \caption{Informazioni sul knowledge graph del dataset LFM1b\_artist}
    \label{tab:dataset_info}
\end{table}

\noindent I nomi delle relazioni presenti nel knowledge graph sono i seguenti:
\begin{itemize}
    \item music.recording.artist
    \item music.recording.releases
    \item music.recording.producer
    \item music.recording.engineer
    \item music.recording.featured\_artists
    \item music.featured\_artist.recordings
    \item music.release.artist
    \item music.artist.release
\end{itemize}

\subsection{Processing del dataset}

\subsubsection{Descrizione procedimento}

\noindent Il dataset originale risultava essere troppo grande per le risorse a nostra disposizione, dunque è stato opportunamente processato. In paritcolare sono state svolte le seguenti operazioni
\begin{itemize}
    \item \textbf{Filtraggio:} il dataset è stato filtrato eliminando tutte le interazioni in cui erano coinvolti utenti e/o item con meno di 5 interazioni
    \item \textbf{Sampling:} dopo la fase di filtraggio, è stato effettuato un sampling casuale il cui scopo era quello di ridurre il numero di interazioni. In particolare sono stati selezionati casualmente 20000 utenti e 50000 item e sono state mantenute solo le interazioni in cui erano coinvolti utenti e item selezionati
\end{itemize}

\noindent In questo modo è stato ottenuto un dataset più piccolo e più facilmente gestibile rispetto a quello originale.
Per poter lavorare su più dataset si è deciso di effettuare un ulteriore processing del dataset, andando a creare dei sampling con una strategia di stratificazione: \footnote{{{Mantenendo il numero di utenti inalterato per ognuno di essi sono stati campionati casualmente un determinato numero di interazioni cercando di mantenere inalterati i "rapporti originali" tra i diversi utenti}}}{}
\begin{itemize}
    \item \textbf{75\%:} Per ogni utente sono state mantenute il 75\% delle interazioni originali
    \item \textbf{50\%:} Dal dataset al 75\% sono state mantenute circa il 66.67\% delle interazioni di ogni utente, in modo tale da avere il 50\% delle interazioni originali
    \item \textbf{25\%:} Dal dataset al 50\% sono state mantenute il 50\% delle interazioni di ogni utente, in modo tale da avere il 25\% delle interazioni originali
\end{itemize}

\subsubsection{Dataset core5}

Descrizione del dataset
\begin{table}[H]
    \centering
    \footnotesize
    \begin{tabularx}{\textwidth}{|c|X|}
        \hline
        \textbf{Feature} & \textbf{Descrizione} \\
        \hline
        n\_users & 120175 \\
        \hline
        n\_items & 585095 \\
        \hline
        n\_inter & 61534450 \\
        \hline
        sparsity & 0.9991248594539152 \\
        \hline
        avg\_inter\_user & 512.0403578115248 \\
        \hline
    \end{tabularx}
    \caption{Informazioni sul dataset LFM1b\_artist\_core5}
    \label{tab:dataset_info}
\end{table}


\noindent Descrizione del knowledge graph
\begin{table}[H]
    \centering
    \footnotesize
    \begin{tabularx}{\textwidth}{|c|X|}
        \hline
        n\_ent\_head & 823213 \\
        \hline
        n\_ent\_tail & 353607 \\
        \hline
        n\_rel & 8 \\
        \hline
        n\_triple & 2114049 \\
        \hline
    \end{tabularx}
    \caption{Informazioni sul knowledge graph del dataset LFM1b\_artist\_core5}
    \label{tab:dataset_info}
\end{table}

\noindent I nomi delle relazioni presenti nel knowledge graph sono i seguenti:
\begin{itemize}
    \item music.recording.artist
    \item music.recording.releases
    \item music.recording.producer
    \item music.recording.engineer
    \item music.recording.featured\_artists
    \item music.featured\_artist.recordings
    \item music.release.artist
    \item music.artist.release
\end{itemize}



\subsubsection{Dataset 20.000 users, 50.000 items}

Descrizione del dataset
\begin{table}[H]
    \centering
    \footnotesize
    \begin{tabularx}{\textwidth}{|c|X|}
        \hline
        \textbf{Feature} & \textbf{Descrizione} \\
        \hline
        n\_users & 19841 \\
        \hline
        n\_items & 42457 \\
        \hline
        n\_inter & 900212 \\
        \hline
        sparsity & 0.9989313587429705 \\
        \hline
        avg\_inter\_user & 45.371301849705155 \\
        \hline
    \end{tabularx}
    \caption{Informazioni sul dataset LFM1b\_artist\_20U50I}
    \label{tab:dataset_info}
\end{table}


\noindent Descrizione del knowledge graph
\begin{table}[H]
    \centering
    \footnotesize
    \begin{tabularx}{\textwidth}{|c|X|}
        \hline
        n\_ent\_head & 15509 \\
        \hline
        n\_ent\_tail & 35156 \\
        \hline
        n\_rel & 5 \\
        \hline
        n\_triple & 46827 \\
        \hline
    \end{tabularx}
    \caption{Informazioni sul knowledge graph del dataset LFM1b\_artist\_20U50I}
    \label{tab:dataset_info}
\end{table}

\noindent I nomi delle relazioni presenti nel knowledge graph sono i seguenti:
\begin{itemize}
    \item music.recording.artist
    \item music.recording.releases
    \item music.recording.producer
    \item music.recording.engineer
    \item music.recording.featured\_artists
\end{itemize}



\subsubsection{Dataset 75\%}

Descrizione del dataset
\begin{table}[H]
    \centering
    \footnotesize
    \begin{tabularx}{\textwidth}{|c|X|}
        \hline
        \textbf{Feature} & \textbf{Descrizione} \\
        \hline
        n\_users & 19841 \\
        \hline
        n\_items & 38932 \\
        \hline
        n\_inter & 667850 \\
        \hline
        sparsity & 0.9991354130849345 \\
        \hline
        avg\_inter\_user & 33.660097777329774 \\
        \hline
    \end{tabularx}
    \caption{Informazioni sul dataset LFM1b\_artist\_20U50I\_75strat}
    \label{tab:dataset_info}
\end{table}


\noindent Descrizione del knowledge graph
\begin{table}[H]
    \centering
    \footnotesize
    \begin{tabularx}{\textwidth}{|c|X|}
        \hline
        n\_ent\_head & 14327 \\
        \hline
        n\_ent\_tail & 32981 \\
        \hline
        n\_rel & 5 \\
        \hline
        n\_triple & 43559 \\
        \hline
    \end{tabularx}
    \caption{Informazioni sul knowledge graph del dataset LFM1b\_artist\_20U50I\_75strat}
    \label{tab:dataset_info}
\end{table}

\noindent I nomi delle relazioni presenti nel knowledge graph sono i seguenti:
\begin{itemize}
    \item music.recording.artist
    \item music.recording.releases
    \item music.recording.producer
    \item music.recording.engineer
    \item music.recording.featured\_artists
\end{itemize}



\subsubsection{Dataset 50\%}

Descrizione del dataset
\begin{table}[H]
    \centering
    \footnotesize
    \begin{tabularx}{\textwidth}{|c|X|}
        \hline
        \textbf{Feature} & \textbf{Descrizione} \\
        \hline
        n\_users & 19841 \\
        \hline
        n\_items & 33653 \\
        \hline
        n\_inter & 440620 \\
        \hline
        sparsity &  0.9993401019218887 \\
        \hline
        avg\_inter\_user & 22.20755002268031 \\
        \hline
    \end{tabularx}
    \caption{Informazioni sul dataset LFM1b\_artist\_20U50I\_50strat}
    \label{tab:dataset_info}
\end{table}


\noindent Descrizione del knowledge graph
\begin{table}[H]
    \centering
    \footnotesize
    \begin{tabularx}{\textwidth}{|c|X|}
        \hline
        n\_ent\_head & 12522 \\
        \hline
        n\_ent\_tail & 29509 \\
        \hline
        n\_rel & 5 \\
        \hline
        n\_triple & 38491 \\
        \hline
    \end{tabularx}
    \caption{Informazioni sul knowledge graph del dataset LFM1b\_artist\_20U50I\_50strat}
    \label{tab:dataset_info}
\end{table}

\noindent I nomi delle relazioni presenti nel knowledge graph sono i seguenti:
\begin{itemize}
    \item music.recording.artist
    \item music.recording.releases
    \item music.recording.producer
    \item music.recording.engineer
    \item music.recording.featured\_artists
\end{itemize}



\subsubsection{Dataset 25\%}

Descrizione del dataset
\begin{table}[H]
    \centering
    \footnotesize
    \begin{tabularx}{\textwidth}{|c|X|}
        \hline
        \textbf{Feature} & \textbf{Descrizione} \\
        \hline
        n\_users & 19841 \\
        \hline
        n\_items & 24878 \\
        \hline
        n\_inter & 218457 \\
        \hline
        sparsity & 0.9995574249320202 \\
        \hline
        avg\_inter\_user & 11.01038254120256 \\
        \hline
    \end{tabularx}
    \caption{Informazioni sul dataset LFM1b\_artist\_20U50I\_25strat}
    \label{tab:dataset_info}
\end{table}


\noindent Descrizione del knowledge graph
\begin{table}[H]
    \centering
    \footnotesize
    \begin{tabularx}{\textwidth}{|c|X|}
        \hline
        n\_ent\_head & 9444 \\
        \hline
        n\_ent\_tail & 23463 \\
        \hline
        n\_rel & 5 \\
        \hline
        n\_triple & 29822 \\
        \hline
    \end{tabularx}
    \caption{Informazioni sul knowledge graph del dataset LFM1b\_artist\_20U50I\_25strat}
    \label{tab:dataset_info}
\end{table}

\noindent I nomi delle relazioni presenti nel knowledge graph sono i seguenti:
\begin{itemize}
    \item music.recording.artist
    \item music.recording.releases
    \item music.recording.producer
    \item music.recording.engineer
    \item music.recording.featured\_artists
\end{itemize}