\begin{frame}{Recommender Systems - Introduzione}
\begin{itemize}
\item Software che suggerisce all'utente elementi di interesse basandosi sulle preferenze e i comportamenti passati.
    \item Basati su Intelligenza Artificiale.
\end{itemize}
\begin{figure}[h!]
    \centering
    \begin{minipage}{0.15\textwidth}
        \centering
        \includegraphics[width=\textwidth]{images/netflix.png}
    \end{minipage}\hfill
    \begin{minipage}{0.15\textwidth}
        \centering
        \includegraphics[width=\textwidth]{images/amazon.png}
    \end{minipage}\hfill
    \begin{minipage}{0.15\textwidth}
        \centering
        \includegraphics[width=\textwidth]{images/spotify.png}
    \end{minipage}\hfill
    \begin{minipage}{0.15\textwidth}
        \centering
        \includegraphics[width=\textwidth]{images/tiktok.png}
    \end{minipage}
    \caption{Alcuni famose piattaforme che utilizzano sistemi di raccomandazione}
\end{figure}
\end{frame}

\begin{frame}{Recommender Systems - Tipologie}
\begin{itemize}
    \item \textbf{Collaborative Filtering}: basato sulle preferenze degli utenti
    \item \textbf{Content-based Filtering}: basato sul contenuto degli item.
    \item \textbf{Knowledge-aware}: utilizzano conoscenza esterna (es. knowledge graph)
    \item \textbf{Hybrid}: combinazione delle precedenti.
\end{itemize}
\end{frame}

\begin{frame}{Recommender Systems -  Alcuni problemi}
\begin{itemize}
    \item Cold Start: difficoltà nel suggerire item a nuovi utenti.
    \item More of the same: suggerire sempre item simili.
    \item \textbf{Sostenibilità: Ad oggi i sistemi di raccomandazione sono dei modelli \textcolor{red}{Red AI}}
\end{itemize}

\textcolor{white}{\textbf{.}}\\
\textcolor{white}{\textbf{.}}\\
\textcolor{white}{\textbf{.}}\\
\textbf{DOMANDA}\\
E' possibile migliorare la sostenibilità di un sistema di raccomandazione?
\end{frame}
