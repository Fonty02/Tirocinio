\begin{frame}{Recommender Systems - Introduzione}
\begin{itemize}
\item Software che suggerisce all'utente elementi di interesse basandosi sulle preferenze e i comportamenti passati.
    \item Basati su Intelligenza Artificiale.
\end{itemize}
\begin{figure}[h!]
    \centering
    \begin{minipage}{0.15\textwidth}
        \centering
        \includegraphics[width=\textwidth]{images/netflix.png}
    \end{minipage}\hfill
    \begin{minipage}{0.15\textwidth}
        \centering
        \includegraphics[width=\textwidth]{images/amazon.png}
    \end{minipage}\hfill
    \begin{minipage}{0.15\textwidth}
        \centering
        \includegraphics[width=\textwidth]{images/spotify.png}
    \end{minipage}\hfill
    \begin{minipage}{0.15\textwidth}
        \centering
        \includegraphics[width=\textwidth]{images/tiktok.png}
    \end{minipage}
    \caption{Alcuni famose piattaforme che utilizzano sistemi di raccomandazione}
\end{figure}
\end{frame}

\begin{frame}{Recommender Systems - Tipologie}
\begin{itemize}
    \item \emoji{handshake} \textbf{Collaborative Filtering}: basato sulle preferenze degli utenti
    \item \emoji{page-facing-up} \textbf{Content-based Filtering}: basato sulla descrizione del contenuto degli item.
    \item \emoji{globe-with-meridians} \textbf{Knowledge-aware}: utilizzano conoscenza esterna (es. knowledge graph)
    \item \emoji{shuffle-tracks-button} \textbf{Hybrid}: combinazione delle precedenti.
\end{itemize}
\end{frame}

\begin{frame}{Recommender Systems -  Sostenibilità}
    Ad oggi i sistemi di raccomandazione non considerano l'impatto ambientale in fase di addestramento, sono dunque modelli \textcolor{red}{Red AI} che puntano alle massime prestazioni senza considerare l'impatto ambientale in termini di CO2.\\
    \textcolor{white}{\textbf{.}}\\
\textcolor{white}{\textbf{.}}\\
\textcolor{white}{\textbf{.}}\\
    \textbf{DOMANDA}\\
    E' possibile migliorare la sostenibilità di un sistema di raccomandazione migliorando il trade-off tra prestazioni e emissioni di CO2?
\end{frame}
