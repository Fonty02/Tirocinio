\begin{frame}{Dataset e modelli utilizzati}
    \footnotesize
    \textbf{Dataset}
    \begin{table}[H]
        \centering
        \begin{tabular}{|c|c|c|c|}
        \hline
        \textbf{Nome} & \textbf{Utenti} & \textbf{Item} & \textbf{Preferenze} \\ \hline
        \emoji{clapper} \textbf{MovieLens10M} & 69,878 & 10,677 film & 10,000,054 \\ \hline
        \emoji{clapper} \textbf{MovieLens1M} & 6,040 & 3,706 film & 1,000,209 \\ \hline
        \emoji{notes} \textbf{LastFM} & 120,322 & 3,123,496 canzoni & 65,133,026 \\ \hline
        \end{tabular}
        \caption{Descrizione dataset}
        \label{tab
        }
        \end{table}
    *I dataset sono stati ridotti selezionando un numero limitato di utenti e di item a causa delle limitazioni dell'infrastruttura utilizzata.\\
    \textbf{Modelli}
    \begin{itemize}
        \item \textbf{Modelli di raccomandazione Collaborative Filtering}: BPR, DMF, LINE, MultiDAE, LightGCN, ItemKNN,NFCF, DGCF
        \item \textbf{Modelli di raccomandazione Knowledge-Aware}: CKE, KGCN, KGNNLS, CFKG
    \end{itemize}
\end{frame}

\begin{frame}{Come sono stati valutati i modelli?}
    \begin{itemize}
        \item \emoji{chart-with-upwards-trend} \textbf{Recall}: capacità di raccomandare item rilevanti
        \item \emoji{chart-with-upwards-trend} \textbf{NDCG}: considera l'ordine degli item raccomandati
        \item \emoji{chart-with-downwards-trend} \textbf{Average Popularity}: misura quanto sono popolari in media gli item raccomandati
        \item \emoji{chart-with-downwards-trend} \textbf{Gini Index}: misura la distribuzione degli item raccomandati
    \end{itemize}
    \end{frame}
    