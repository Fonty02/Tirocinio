\makesection{Sostenibilità}

\begin{frame}{Sostenibilità} 
        La sostenibilità è un concetto complesso e multidimensionale, emerso negli anni '80, che ha acquisito un'importanza crescente nella società contemporanea. Essa si riferisce alla capacità di soddisfare i bisogni presenti senza compromettere quelli delle generazioni future, coinvolgendo la gestione responsabile delle risorse naturali, la tutela ambientale, lo sviluppo economico e sociale e la garanzia di un futuro migliore.
        La sostenibilità ambientale è uno degli aspetti più importanti della sostenibilità e riguarda la gestione responsabile
        delle risorse naturali, la tutela dell’ambiente e la prevenzione dell’inquinamento e del degrado ambientale. 
\end{frame}

\begin{frame}{Sostenibilità e AI} 
Nell'ambito dell'Intelligenza Artificiale (AI) e della sostenibilità ambientale, possiamo distinguere due tipi di AI sostenibile:
\begin{itemize}
\item \textbf{Sustainability of AI}: si concentra sulla misurazione della sostenibilità nello sviluppo e nell'uso dei modelli AI, come la carbon footprint e l'energia necessaria per addestrarli.
\item \textbf{AI for Sustainability}: utilizza l'AI per affrontare le sfide della sostenibilità, come la previsione del cambiamento climatico e la gestione delle risorse naturali.
\end{itemize}
La \textcolor{green}{Green AI} si riferisce allo sviluppo di modelli AI che considerano il costo computazionale e l'impatto ambientale. In contrasto, la \textcolor{red}{Red AI} mira a creare modelli sempre più complessi senza considerare le risorse impiegate. La Green AI riduce parametri, complessità e operazioni computazionali per minimizzare l'uso di risorse energetiche.
\end{frame}
