\begin{frame}{Addestramento sostenibile - Introduzione}
    \begin{wrapfigure}{r}{0.6\textwidth}
        \centering
        \includegraphics[width=0.6\textwidth, height=0.6\textheight]{images/curve_emissions_score.png}
        \caption{Andamento score e emissioni}
    \end{wrapfigure} 
Sull'asse delle x troviamo le emissioni, sull'asse delle y lo score. Quando la derivata è al di sotto di una certa soglia \textbf{S} per un certo numero di epoche consecutive \textbf{E} l'addestramento termina (comportamento asintotico).
Approssimazione della derivata della curva:
\begin{equation*}
    \frac{f(x_{i+1}) - f(x_i)}{x_{i+1} - x_i}
\end{equation*}
\end{frame}

\begin{frame}{Addestramento sostenibile - Esplorazione}
    \begin{table}[]
        \centering
        \begin{tabular}{|c|c|c|c|}
            \hline
            \textbf{Esperimento} & \textbf{Dataset} & \textbf{Soglia} & \textbf{Epoche} \\ \hline
            Esperimento 1 & MovieLens1M & 50 & 5 \\ \hline
            Esperimento 2 & LastFM & 30 & 7 \\ \hline
            Esperimento 3 & Amazon\_Books & 40 & 6 \\ \hline
        \end{tabular}
        \caption{Parametri degli Esperimenti}
    \end{table}
    \begin{itemize}
        \item \textbf{Conclusioni:} Alcuni modelli (es. DGCF) sono molto sensibili al nuovo criterio, altri (es. DMF) meno
    \end{itemize}
\end{frame}


\begin{frame}{Addestramento sostenibile - Confronto criteri}
    Fissato il dataset MovieLens1M, sono stati eseguiti 6 esperimenti variando i parametri:
    \begin{table}[]
        \centering
        \begin{tabular}{|c|c|c|}
            \hline
            \textbf{Esperimento} & \textbf{Soglia} & \textbf{Epoche} \\ \hline
            Esperimento 1 & 40 & 5 \\ \hline
            Esperimento 2 & 30 & 5 \\ \hline
            Esperimento 3 & 40 & 6 \\ \hline
            Esperimento 4 & 30 & 6 \\ \hline
            Esperimento 5 & 40 & 7 \\ \hline
            Esperimento 6 & 30 & 7 \\ \hline
        \end{tabular}
        \caption{Parametri degli Esperimenti}
    \end{table}
\end{frame}


\begin{frame}{Addestramento sostenibile - Risultati confronto criteri}
\begin{table}[H]
    \scriptsize
    \centering
    \begin{tabular}{|c|c|c|}
        \hline
        \textbf{Modello} & \textbf{Parametro più impattante} & \textbf{Migliori risultati} \\
        \hline
        BPR & Soglia & Soglia 40 e 6 epoche \\
        \hline
        CFKG & Soglia & Soglia 40 e 6 epoche \\
        \hline
        CKE & Epoche consecutive & Soglia 40 e 6 epoche \\
        \hline
        DMF & Nessuno predominante & Soglia 40 e 7 epoche \\
        \hline
        KGCN & Epoche consecutive & Soglia 40 e 5 epoche \\
        \hline
        KGNNLS & Soglia & Soglia 40 e 5 epoche \\
        \hline
        LINE & Soglia & Soglia 40 e 7 epoche \\
        \hline
        MultiDAE & Soglia & Soglia 40 e 7 epoche \\
        \hline
        LightGCN & Soglia & Soglia 40 e 6 epoche \\
        \hline
        NGCF & Epoche consecutive & Soglia 40 e 5 epoche \\
        \hline
        DGCF & Epoche consecutive & Soglia 40 e 6 epoche \\
        \hline
    \end{tabular}
    \caption{Parametri più impattanti e migliori risultati per ciascun modello}
\end{table}
\end{frame}


\begin{frame}{Addestramento sostenibile - Risultati confronto criteri}
\begin{table}[H]
    \scriptsize
    \centering
        \begin{tabular}{|c|c|c|c|}
            \hline
            \textbf{Tipo di Modello} & \textbf{Parametro predominante} & \textbf{Numero di Modelli} & \textbf{Modelli} \\ \hline
            Collaborative Filtering & Soglia & 5 & BPR, DMF, LightGCN, MultiDAE, LINE \\ \hline
            Collaborative Filtering & Epoche & 2 & NGCF, DGCF \\ \hline
            Knowledge Aware & Soglia & 2 & CFKG, KGNNLS \\ \hline
            Knowledge Aware & Epoche & 2 & CKE, KGCN \\ \hline
        \end{tabular}
    \caption{Riassunto dei parametri dominanti per tipo di modello}
\end{table}

\end{frame}