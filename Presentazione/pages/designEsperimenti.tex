\begin{frame}{Domande di ricerca e lavoro svolto}
    \begin{itemize}
        \item \textbf{RQ1}: Qual è il trade-off tra emissioni e performance dei modelli di raccomandazione a stato dell'arte\footnote{Modelli classici a cui fare riferimento}{?}
        \item \textbf{RQ2}: E’ possibile usare un criterio di early-stopping basato sulle emissioni per migliorare il trade-off tra emissioni e performance dei modelli di raccomandazione a stato dell’arte?
        \item \textbf{RQ3}: Quali parametri possono essere utilizzati in questi criteri per migliorare il trade-off?
    \end{itemize}
\end{frame}

\begin{frame}{CodeCarbon}
    Per il tracking delle emissioni è stata usata la libreria Python \textbf{CodeCarbon}, la quale usa l'equivalente di anidride carbonica (CO$_2$eq) per misurare le emissioni mediante la seguente formula.\\
    \begin{equation}
        \textit{emission} = \textit{CI}  \cdot \textit{PC}
    \end{equation} dove \textit{CI} è il Carbon Intensity e \textit{PC} è il Power Consumption (cioè l'energia consumata).\\
    
    I valori di CI dipendono dalle diverse fonti di energia utilizzate durante la computazione 
    (es. energia solare, energia eolica, etc.). Se \textit{s} è la fonte di energia,  \textit{e$_s$} sono le emissioni per KW/h di energia e \textit{p$_s$}  è la percentuale di energia prodotta dalla fonte s, allora il CI è dato da:
    \begin{equation}
        \textit{CI} = \sum_{s \in S} \textit{e$_s$} \cdot \textit{p$_s$}
    \end{equation}
    
\end{frame}
