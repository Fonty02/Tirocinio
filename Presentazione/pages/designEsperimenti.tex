\begin{frame}{Reserch Questions}
    \begin{itemize}
        \item \textbf{RQ1}: Qual è il trade-off tra emissioni e performance dei modelli di raccomandazione a stato dell'arte\footnote{Modelli classici a cui fare riferimento}{?}
        \item \textbf{RQ2}: E’ possibile usare un criterio di early-stopping basato sulle emissioni per migliorare il trade-off tra emissioni e performance dei modelli di raccomandazione a stato dell’arte?
        \item \textbf{RQ3}: Quali parametri possono essere utilizzati in questi criteri per migliorare il trade-off?
    \end{itemize}
\end{frame}

\begin{frame}{Lavoro svolto}
    Per rispondere alle domande di ricerca sono state svolte le seguenti attività:
    \begin{itemize}
        \item \textbf{Benchmarking}: Addestramento di modelli di raccomandazione e misurazione delle emissioni
        \item \textbf{Addestramento sostenibile}: Studio del criterio di early-stopping
        \item \textbf{Parametri di miglioramento}: Studio di parametri per migliorare il trade-off
    \end{itemize}
    Sono state utilizzate le librerie \textbf{RecBole} e \textbf{CodeCarbon}.

    \begin{equation*}
        \textit{emission} = \textit{CI}  \cdot \textit{PC}
    \end{equation*}
    
    \begin{equation*}
        \textit{CI} = \sum_{s \in S} \textit{e$_s$} \cdot \textit{p$_s$}
    \end{equation*}
    
\end{frame}