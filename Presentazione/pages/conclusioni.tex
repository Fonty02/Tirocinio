\makesection{Conclusioni e sviluppi futuri}
\begin{frame}{Conclusioni}
    \begin{block}{Benchmarking}
        Vengono confermate le ipotesi iniziali per cui spesso i modelli più complessi hanno emissioni maggiori non giustificate da un miglioramento delle performance elevato.
    \end{block}
    \begin{block}{Regressore}
        Il nuovo dataset è più ricco del precedente ma anche più sbilanciato
    \end{block}
    \begin{block}{Addestramento sostenibile}
        E' possibile ridurre le emissioni di un modello di raccomandazione senza perdere in modo significativo di performance
    \end{block}
\end{frame}



\begin{frame}{Sviluppi futuri}
    \scriptsize
    \begin{block}{Benchmarking}
        E’ necessario effettuare più esperimenti variando dataset,modelli e hardware per avere una visione più completa del problema.
    \end{block}
    \begin{block}{Regressore}
        Con più dati a disposizione si potrebbero creare modelli più complessi (come reti neurali) per cercare di migliorare le performance.
    \end{block}
    \begin{block}{Addestramento sostenibile}
        Eseguire più esperimenti con altri dataset e altri hardware per confermare o meno i risultati ottenuti.
    \end{block}
    \begin{block}{Iperparametri}
        Tutti gli esperimenti sono stati effettuati con iperparametri di default. Dunque tutta la fase di benchmarking e di addestramento sostenibile potrebbe essere rivista anche in termini di ricerca degli iperparametri migliori.
    \end{block}
\end{frame}
